%%%%%%%%%%%%%%%%%%%%%%%%%%%%%%%%%%%%%%%%%%%%%%%%%%%%%%%%%%%%%%%%%%%%%%%%%%%%%%%%
% TUM-Vorlage: Wissenschaftliche Arbeit
%%%%%%%%%%%%%%%%%%%%%%%%%%%%%%%%%%%%%%%%%%%%%%%%%%%%%%%%%%%%%%%%%%%%%%%%%%%%%%%%
%
% Rechteinhaber:
%     Technische Universität München
%     https://www.tum.de
% 
% Gestaltung:
%     ediundsepp Gestaltungsgesellschaft, München
%     http://www.ediundsepp.de
% 
% Technische Umsetzung:
%     eWorks GmbH, Frankfurt am Main
%     http://www.eworks.de
%
%%%%%%%%%%%%%%%%%%%%%%%%%%%%%%%%%%%%%%%%%%%%%%%%%%%%%%%%%%%%%%%%%%%%%%%%%%%%%%%%

%%%%%%%%%%%%%%%%%%%%%%%%%%%%%%%%%%%%%%%%%%%%%%%%%%%%%%%%%%%%%%%%%%%%%%%%%%%%%%%%
\input{./Ressourcen/Praeambel.tex} % !!! NICHT ENTFERNEN !!!
%%%%%%%%%%%%%%%%%%%%%%%%%%%%%%%%%%%%%%%%%%%%%%%%%%%%%%%%%%%%%%%%%%%%%%%%%%%%%%%%

\renewcommand{\Thema}{%
    Thema der Arbeit (optional)}

%%%%%%%%%%%%%%%%%%%%%%%%%%%%%%%%%%%%%%%%%%%%%%%%%%%%%%%%%%%%%%%%%%%%%%%%%%%%%%%%
\input{./Ressourcen/Anfang.tex} % !!! NICHT ENTFERNEN !!!
%%%%%%%%%%%%%%%%%%%%%%%%%%%%%%%%%%%%%%%%%%%%%%%%%%%%%%%%%%%%%%%%%%%%%%%%%%%%%%%%

\begin{document}

\title{Thema der Arbeit}
\author{Martin Mustermann}
\date{Datum}


\tableofcontents % Inhaltsverzeichnis

\chapter{Gripper Control: Impedance And Admittance Control}

\section{Control Structure}

%\subsection[]{Absatzüberschrift}



\begin{figure}[!ht]
\centering
\noindent\hspace{0.5mm}\includegraphics[width=12cm]{/home/chenshengchen/Downloads/RoboCup@Home/Thesis_templates/Thesis/ICS1.png}\\
%\caption{Titel, Autor}
\centering
Figure 1.1\\
\end{figure}

\section{Control Law}

\begin{figure}[!ht]
\centering
\noindent\hspace{0.5mm}\includegraphics[width=12cm]{/home/chenshengchen/Downloads/RoboCup@Home/Thesis_templates/Thesis/ICS3.png}\\
%\caption{Titel, Autor}
\centering
Figure 1.2\\
\end{figure}

Designed property:\quad M$\Delta\ddot x+B\Delta\dot x+K\Delta x=F_{ext}$\\
\vspace{1 ex}
Target state:$\quad \Delta \ddot x_d = 0 \quad \Delta \dot x_d = 0\quad x_d = x_d$\\
\vspace{1 ex}
Here, two target states: $``grip": x_d = 0$,\ $``release": x_d = 0.04$\\
\vspace{1 ex}
Using accelreation to control the gripper, setting current state $\ddot x_c,  \dot x_c, x_c$\\
\vspace{1 ex}
$\Rightarrow$ Control function:\quad M$\ddot x_c+B\dot x_c+K(x_c-x_d)=F_{ext}$

\section{Pactical Problem}
Problem:\ Tiago can't control using $\ddot x$ or force\\
\vspace{1 ex}
Solution:\ Using $\ddot x$ to calculate $\dot x$ and $x$\\
\vspace{1 ex}
$\hspace{1cm} \quad \ \dot x{(t+dt)} = \dot x(t) + \ddot x(t+dt)*dt$

\clearpage

Passen Sie gegebenenfalls die Ränder an die Vorgaben Ihres Prüfers an und
beachten Sie dabei, dass das Logo der TUM sich oben rechts innerhalb der
Ränder, auf der Titelseite befindet. Für die Titelseiten stehen separate
Vorlagen zur Verfügung.

Zur Definition von \gls{abk} erstellen Sie für die gewünschte Abkürzung einen
Eintrag in der Datei \texttt{Abkuerzungen.tex} und referenzieren sie ihn
mittels \texttt{\textbackslash{}gls}; diese tauchen nach einem Lauf mit
\texttt{latexmk} im Abkürzungsverzeichnis auf. Beispiel:

\vspace{-\baselineskip}
\begin{description}[leftmargin=1em+5mm, labelindent=5mm]
\item[Definition in \texttt{Abkuerzungen.tex}:] \texttt{\textbackslash{}newacronym\{abk\}\{Abk.\}\{Abkürzungen\}\}}
\item[Referenzierung:] \texttt{\textbackslash{}gls\{abk\}}
\end{description}

Für weitere Informationen zu Glossaren und Abkürzungen siehe die Dokumentation
des Pakets \texttt{glossaries} und die entsprechenden Abschnitte in den
Vorlagendateien.


\subsection[]{Aufzählungen}

\begin{itemize}
\item Dies ist die Standardaufzählung
    \begin{itemize}
    \item Dies ist die nächste Ebene der Aufzählung
    \end{itemize}
\end{itemize}


\subsection[]{Nummerierungen}

\begin{enumerate}
\item Erster Punkt der Nummerierungen
    \begin{enumerate}
    \item Unterpunkt der Nummerierungen
    \end{enumerate}
\end{enumerate}
\clearpage

\listoffigures % Abbildungsverzeichnis

\printacronyms[title={Abkürzungsverzeichnis}] % Abkürzungsverzeichnis

\listoftables % Tabellenverzeichnis

\onehalfspacing

\addchap{Tabellenvarianten}

\vspace{22mm}
\section*{Überschrift Tabelle 1}

\begin{table}[!h]
\begin{tabularx}{\textwidth + 5pt}{@{\hspace{3pt}} M | @{\hspace{3pt}} M}
\multicolumn{2}{@{}X}{%
    \begin{tabularx}{\textwidth}{@{\hspace{3pt}} M @{\hspace{14.5pt}} M}
    \textbf{Spalte 1} & \textbf{Spalte 2}
    \end{tabularx}%
} \\
\hline
Nummer 1 & Nummer 2 \\
\hline
Nummer 1 & Nummer 2 \\
\hline
Nummer 1 & Nummer 2 \\
\hline
\end{tabularx}

\caption{Beschreibung}
\end{table}


\vspace{\parskip}
\section*{Überschrift Tabelle 2}

\begin{table}[!h]
\hspace{-5pt}
\begin{tabularx}{\textwidth + 5pt}{| @{\hspace{3pt}} M | @{\hspace{3pt}} M |}
\hline
\textbf{Spalte 1} & \textbf{Spalte 2} \\
\hline
Nummer 1 & Nummer 2 \\
\hline
Nummer 1 & Nummer 2 \\
\hline
Nummer 1 & Nummer 2 \\
\hline
\end{tabularx}
\caption{}
\end{table}


\vspace{\parskip}
\section*{Überschrift Tabelle 3}

\begin{table}[!h]
\begin{tabularx}{\textwidth}{@{} M M}
\textbf{Spalte 1} & \textbf{Spalte 2} \\
Nummer 1 & Nummer 2 \\
Nummer 1 & Nummer 2 \\
Nummer 1 & Nummer 2 \\
\end{tabularx}
\caption{}
\end{table}

\clearpage

\addchap{Tabellenvarianten 2}

\vspace{22mm}
\section*{Überschrift Tabelle 1}

\begin{table}[!h]
\fontsize{9pt}{13pt}\selectfont
%\renewcommand{\arraystretch}{1.8}
\hspace{-5pt}
\begin{tabularx}{\textwidth + 5pt}{@{\hspace{3pt}} M | @{\hspace{3pt}} M}
\multicolumn{2}{@{}X}{%
    \begin{tabularx}{\textwidth}{@{\hspace{3pt}} M @{\hspace{14.5pt}} M}
    \textbf{Spalte 1} & \textbf{Spalte 2}
    \end{tabularx}%
} \\
\hline
Nummer 1,\newline\,mehrzeilig in Schriftgröße 9 pt & Nummer 2 \\
\hline
Nummer 1 & Nummer 2 \\
\hline
Nummer 1 & Nummer 2 \\
\hline
\end{tabularx}

\caption{}
\end{table}


\vspace{\parskip}
\section*{Überschrift Tabelle 2}

\begin{table}[!h]
\fontsize{9pt}{13pt}\selectfont
\hspace{-5pt}
%\renewcommand{\arraystretch}{1.8}
\begin{tabularx}{\textwidth + 5pt}{| @{\hspace{3pt}} M | @{\hspace{3pt}} M |}
\hline
\textbf{Spalte 1} & \textbf{Spalte 2} \\
\hline
Nummer 1 & Nummer 2 \\
\hline
Nummer 1 & Nummer 2 \\
\hline
Nummer 1 & Nummer 2 \\
\hline
\end{tabularx}
\caption{}
\end{table}


\vspace{\parskip}
\section*{Überschrift Tabelle 3}

\begin{table}[!h]
\fontsize{9pt}{13pt}\selectfont
%\renewcommand{\arraystretch}{1.8}
\begin{tabularx}{\textwidth}{@{} M M}
\textbf{Spalte 1} & \textbf{Spalte 2} \\
Nummer 1 & Nummer 2 \\
Nummer 1 & Nummer 2 \\
Nummer 1 & Nummer 2 \\
\end{tabularx}
\caption{}
\end{table}

\end{document}